%%%%%%%%%%%%%%%%%%%%%%%%%%%%%%%%%%%%%%%%%%%%%%%%%%%%%%%%%%%%%%%%%%%%%%%%%%%%%%%%
%2345678901234567890123456789012345678901234567890123456789012345678901234567890
%        1         2         3         4         5         6         7         8

\documentclass[letterpaper, 10 pt, conference]{ieeeconf}  % Comment this line out
                                                          % if you need a4paper
%\documentclass[a4paper, 10pt, conference]{ieeeconf}      % Use this line for a4
                                                          % paper

\overrideIEEEmargins
% See the \addtolength command later in the file to balance the column lengths
% on the last page of the document

\usepackage[utf8]{inputenc}
\usepackage[T1]{fontenc}

\title{\LARGE \bf
Developer Industry Readiness
}

\author{John Carmack - Dylan Lomax - Lillian Coar - IDK}% <-this % stops a space


\begin{document}



\maketitle
\thispagestyle{empty}
\pagestyle{empty}


%%%%%%%%%%%%%%%%%%%%%%%%%%%%%%%%%%%%%%%%%%%%%%%%%%%%%%%%%%%%%%%%%%%%%%%%%%%%%%%%
\begin{abstract}

With the growing importance of formal education for developers to find employment, an
obvious question is whether or not that formal education actually prepares the developer 
for the workforce. We explore that question by asking if the programming language developers 
are taught in their programs are actually in demand and are present in their work lives.

\end{abstract}


%%%%%%%%%%%%%%%%%%%%%%%%%%%%%%%%%%%%%%%%%%%%%%%%%%%%%%%%%%%%%%%%%%%%%%%%%%%%%%%%
\section{Objective}

It is our goal to answer the question of whether or not a developer's first language is indicative of their success in the software engineering workplace. We wish to ascertain whether or not the programmer's choice of beginning language, or the language that their curriculum is taught in has any impact whatsoever in their career prospects, or in their ability to adapt and assimilate to a working environment typical to the modern software engineer, which seems incredibly saturated with positions requesting experience with web technologies and JavaScript frameworks. 

\section{Motivation}

Our motivation for pursuing these questions is three-fold. 
First, as students of computer science who are late in our formal education, the question of our preparedness for the workplace is an immediate one. We have long been sold the idea that the language of our formal training is inconsequential, and we would like to collect the data and analyze it for ourselves to ascertain the validity of such claims. Secondly, a happy consequence of this analysis is that our team will become more conversant in these in-demand technologies and philosophies as we learn how to collect, analyze and scrub large sets of data. Finally, perhaps our findings may illuminate issues within the formal training of computer scientists and software engineers and steps might be taken by the purveyors of such curricula to mitigate gaps in student preparedness. 

\section{Data}

The collection of data to answer these questions will come from a multitude of sources. 

*** ADD STUFF HERE ***
\subsection{Organizational data}
*** ADD STUFF HERE ***
\subsection{Government data}
*** ADD STUFF HERE ***
\subsection{Social Media Polling}
*** ADD STUFF HERE ***

\section{Team Member Responsibilities}
Member responsibilities are relatively flexible to date, with the following guidelines in place for general task assignment. 

\subsection{John Carmack}
Data cleaning
Visualization
Documentation.

\subsection{Dylan Lomax}
*** ADD STUFF HERE ***
\subsection{Lillian Coar}
*** ADD STUFF HERE ***
\subsection{IDK}
*** ADD STUFF HERE ***
 
\section{Timeline}
*** ADD STUFF HERE ***

\section{Expected Outcomes}

*** JOHN'S TAKE. FEEL FREE TO AXE IT *** 

It is reasonable to assume that the formal training computer science students receive is not sufficient to become competitive as an applicant to industry positions, especially in a space dominated by web development positions. The hypothesis is that a student's beginning language is less consequential to success in the workplace employment than the ability to translate that knowledge to other technologies and be able to learn and utilize them rapidly. 

\end{document}

